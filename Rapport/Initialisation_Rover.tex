\documentclass{article}
\usepackage[utf8]{inputenc}

\title{Rover Initialisation}

\author{Clement LEBOUIS, Soline LECOMTE, Vincent KOWALSKI}

\date{January 2020}

\begin{document}

\maketitle

\section{Introduction}
This document will explain how to initialize the LEGO MINDSTORMS EV3 used in this project.
All manipulation will be done on Windows 10, and the interface used is from january 2020.

\section{Eclipse}

Tha Java program is run on Eclipse to send instructions to the rover. You can install Eclipse from this URL :
https://www.eclipse.org/downloads/

After this, you will have to install the LeJOS plugin, which is not (currently) available on market place.
In order to do so, go to "Help" -> "Install New Software" -> "Add".
In the new window, enter :
Name : leJOS EV3
Location :  http://lejos.sourceforge.net/tools/eclipse/plugin/ev3
And click on "Add".

Select "leJOS EV3 Support" in the list and apply modifications.

For the Java code you can use, for exemple, this one :
https://github.com/FrapperColin/2017-2018/tree/master/ProjectRileyRoverLejos/ProjectRileyRover

\section{LeJOS EV3}

To communicate between the computer and the rover, it is also necessary to intall the LeJOS EV3 software.
You can download it at this adress :
https://sourceforge.net/projects/ev3.lejos.p/

Before starting installation, make sure you have java JDK 1.8 installed.
You can download it at this adress :
https://www.oracle.com/technetwork/java/javase/downloads/jdk8-downloads-2133151.html

Then, start your installation : you can follow the quick installation below or the detailed installation later.

Quick installation :

Click on "Next"
Select your Java JDK 1.8 main directory and click "Next"
Click on "Next" -> "Next" -> "Next" -> "Next" -> "Install"
After installation click on "Finish"

Detailed installation :

Click on "Next"
Select your Java JDK 1.8 main directory and click "Next"
Chose where to install software and click "Next"
Chose all additional files to install (none of them is needed for this project, the choice is up to you) and click "Next"
If you want to install additional files, choses where to install them, and click "Next"
Indicate the start menu folder name. If you don't want any, click on the checkbox, then click on "Next"
Click on "Install" and after installation click on "Finish".

\section{SD card configuration}

After the installation, the EV3 SD Card Creator should have been launched.

Start by selecting your SD card with "Select SD drive:"
Then, click on "Click the link to download the EV3 Oracle JRE." to download the JRE to install on the SD card.
Make sure to download the file with an name starting by ejre.... If you take the ejdk... file it will not be recognized.
Click on "JRE" and select the file to download
Click on "Create"
After installation, you can safely eject your SD card.

\end{document}
